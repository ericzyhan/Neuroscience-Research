\documentclass{article}
\usepackage{graphicx} % Required for inserting images
\usepackage[english]{babel} % English formatting
\usepackage[utf8]{inputenc} % Standard encoding
\usepackage[a4paper,left=4cm,bottom=2cm]{geometry} % Page formatting
\usepackage{indentfirst} % Indents the first paragraph
\usepackage{amsmath} % Maths type package
\usepackage{bm} % Bold font maths
\usepackage{graphicx} % Advanced graphics package
\usepackage[export]{adjustbox} 
\usepackage{fancyhdr} % Fancy headers

\usepackage{wrapfig} % Text flowing around figures

\pagestyle{fancy}
\fancyhf{}
\renewcommand{\footrulewidth}{0.4pt}

\title{Random CTLNs}
\author{Eric Han, Caitlin Lienkaemper}
\date{July-August 2024}

\begin{document}

\maketitle

\section{Project Summary}
The goal of this project was to observe the effects of symmetry in random combinatorial linear-threshold networks (CTLNs). The simulation model was built in Python. A major step taken this summer was the completion of a small but complete program to generate averaged heatmaps of how network dynamics behaved across various probability parameters. Additionally, at the tail-end of the project, a more mathematically focused thrust to the problem was conceptualized and a more concrete vision of tackling the symmetry problem was realized.

\section{Model and Code}
\subsection{Algorithm}
The goal of this model is to generate random graphs with set parameters for symmetry and edge connection probability. This is done through the generation of a random square adjancency matrix of size $n$.

\end{document}
